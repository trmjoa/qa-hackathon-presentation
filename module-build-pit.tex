\documentclass[10pt]{beamer}
\usepackage[utf8]{inputenc}
\title{Perl \\ Post Install Tests}

\begin{document}

\begin{frame}
\titlepage
\end{frame}


\begin{frame}[fragile]
\frametitle{Upgrading dependencies w/fear}
\begin{enumerate}
  \item Install a module named \verb|Bear|
  \item Install the module \verb|Human|, which depends on the availability of \verb|Bear|'s \verb|ride()| method
  \item (time passes...)
  \item Upgrade to a newer version of \verb|Bear|, which does NOT have the \verb|ride()| method; The authhor of \verb|Bear| learned it's not safe to ride bears.
  \item *boom* \verb|Human|'s attempt to \verb|ride()| fails!
  \item \verb|Human| gets eaten by \verb|Bear| and/or customer.
\end{enumerate}
\end{frame}

\begin{frame}
\frametitle{Sources of \emph{upgrade fear}}
\begin{itemize}
  \item Don't assume CPAN authors have a commitment to their APIs. They may change at any time - even if the CPAN/Perl community has a history of preserving backwards compatibility!
  \item No \emph{trivial} way of verifying if the installed Perl/CPAN dists still work.
  \item Modules might be used in ways unintended by the author.
\end{itemize}
Detecting an upgrade bug/bear early, is \emph{good}.
\end{frame}

\begin{frame}[fragile]
\frametitle{Detecting breakage}
We have the following options:
\begin{itemize}
\item Run your code and see
\item Run integration/smoke tests for YOUR code
\item Check cpantesters.org
\end{itemize}
\end{frame}

\begin{frame}[fragile]
\frametitle{We want more options for detecting breakage!}
\begin{itemize}
\item Run tests for all packages found in \verb|@INC|
\item Run tests for the distributions that depend on the distribution you're about to upgrade
\item Run tests in any runtime environment (development, test/CI, staging and if necessary, {\bf production})
\end{itemize}
\end{frame}

\begin{frame}
\frametitle{Post Install Tests Requirements}
\begin{itemize}
\item The tests for the installed modules must be available (installed)
\item It must be possible to locate tests for a specific distribution and version
\item In order to run the tests for only the dependent distributions it must be possible to do determine this distribution's reverse dependencies
\item Distribution dependencies are saved and installed
\item Writing a best practices guide for tests
\end{itemize}
\end{frame}

\begin{frame}[fragile]
\frametitle{Demo - Module::Build::PIT}
\begin{itemize}
\item Extension of \verb|Module::Build|
\item Environment variable \verb|PERL_INSTALL_TESTS| controls if tests should be installed when using  \verb|./Build install| (Also possible to use \verb|./Build installtests| without the ENV)
\item Tests is installed under \verb|$install_base/auto/tests/$distname-$distversion/|
\item Test files are written to the \verb|.packlist| file
\item Action \verb|testinc| uses the \verb|ExtUtils::Installed| distribution to retrieve the modules that would be loaded and their test files
\item Action \verb|testrdeps| is a fake action. The logic is hardcoded for demonstration purposes
\end{itemize}
\end{frame}

\begin{frame}[fragile]
\frametitle{Future Work / Considerations}
\begin{itemize}
\item Permissions - Test that writes to devices, sockets(ports \textless \ 1024), t/... and more
\item Currently no trivial way of finding reverse dependencies locally
\item Integration with packaging systems
\item Upgrading \verb|test_requires| to regular \verb|requires|?
\item Best practices doc
\end{itemize}
\end{frame}

\begin{frame}[fragile]
\frametitle{Code}
\begin{itemize}
\item Code is on github!
\item \verb|git clone| \verb|https://github.com/bannaN/Module-Build-PIT.git|
\item \verb|git clone| \verb|https://github.com/bannaN/qa-hackathon-presentation.git|
\item Fork it now!
\end{itemize}
\end{frame}

\end{document}
