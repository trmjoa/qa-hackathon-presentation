\documentclass[10pt]{beamer}
\usepackage[utf8]{inputenc}
\title{Perl \\ Post Install Tests}

\begin{document}

\begin{frame}
\titlepage
\end{frame}


\begin{frame}[fragile]
\frametitle{Upgrading dependencies w/o fear}
\begin{enumerate}
  \item You install a module named \verb|Bear|
  \item You install the module \verb|Human|, which depends on the availability of Bear's \verb|ride()| method
  \item You install an new version of \verb|Bear|, which does NOT have the \verb|ride()| method; it's not safe to ride a bear!
  \item Human's attempt to \verb|ride()| fails
  \item \verb|Human| gets eaten by \verb|Bear|
\end{enumerate}
\end{frame}

\begin{frame}
\frametitle{Sources of \emph{upgrade fear}}
\begin{itemize}
\item We should not assume that CPAN authors have a formal commitment to their APIs, they may change at any time
\item No trivial way of verifying the whole perl installation with distributions
\item CPAN/Perl community has usually been good at preserving backwards compatibility
\item Modules might get used in an unintended way from the authors perspective
\end{itemize}
\end{frame}

\begin{frame}[fragile]
\frametitle{How to detect breakage}
You could do the following and hope that you detect the bug/ breakage:
\begin{itemize}
\item Run your code and see
\item Run tests for YOUR code
\item Check cpantesters.org
\end{itemize}
\end{frame}
\begin{frame}[fragile]
\frametitle{What we want}
\begin{itemize}
\item Run tests for all distributions loadable in \verb|@INC|
\item Run tests for the distributions that depend on the newly upgraded distribution
\item Run tests in all environments (development, test/CI, staging and {\bf production})
\end{itemize}
\end{frame}

\begin{frame}
\frametitle{Post Install Tests Requirements}
\begin{itemize}
\item The tests for the installed modules must be available(installed)
\item It must be possible to locate tests for a specific distribution and version
\item In order to run the tests for only the dependent distributions it must be possible to do determine this distribution's reverse dependencies
\item Distribution dependencies are saved and installed
\item Writing a best practices guide for tests
\end{itemize}
\end{frame}

\begin{frame}[fragile]
\frametitle{Demo}

\begin{itemize}
\item Extension of \verb|Module::Build|
\item Environment variable \verb|PERL_INSTALL_TESTS| controls if tests should be installed when using  \verb|./Build install|(Also possible to use \verb|./Build installtests| without the ENV)
\item Tests is installed under \verb|$install_base/auto/tests/$distname-$distversion/|
\item Test files are written to the \verb|.packlist| file
\item Action \verb|testinc| uses the \verb|ExtUtils::Installed| distribution to retrieve the modules that would be loaded and their test files
\item Action \verb|testrdeps| is a fake action. The logic is hardcoded for demonstration purposes
\end{itemize}
\end{frame}

\begin{frame}[fragile]
\frametitle{Future Work / Considerations}
\begin{itemize}
\item Permissions - Test that writes to devices, sockets(ports \textless \ 1024), t/... and more
\item Currently no trivial way of finding reverse dependencies locally
\item Integration with packaging systems
\item Best practices doc
\end{itemize}
\end{frame}

\end{document}